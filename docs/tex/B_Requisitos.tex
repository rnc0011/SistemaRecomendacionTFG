\apendice{Especificación de Requisitos}

\section{Introducción}
Este anexo recoge los objetivos generales y la especificación de requisitos del proyecto.

\section{Objetivos generales}
El proyecto persigue los siguientes objetivos generales:
\begin{itemize}
\tightlist
\item Comprender los sistemas de recomendación tanto clásicos como basados en aprendizaje profundo.
\item Recoger y evaluar los resultados obtenidos por los dos modelos sobre diferentes conjuntos de datos.
\item Comparar los resultados.
\end{itemize}

\section{Catalogo de requisitos}
Los requisitos derivados de los objetivos del proyecto son los siguientes:
\subsection{Requisitos funcionales}
\begin{itemize}
\tightlist
\item \textbf{RF-1 Gestión de usuarios:} el programa tiene que ser capaz de gestionar los nuevos usuarios.
\begin{itemize}
\tightlist
\item \textbf{RF-1.1 Añadir usuarios:} el programa tiene que ser capaz de añadir nuevos usuarios.
\item \textbf{RF-1.2 Modificar usuarios:} el programa tiene que ser capaz de modificar las características de los usuarios y/o sus valoraciones.
\item \textbf{RF-1.3 Eliminar usuarios:} el programa tiene que ser capaz de eliminar usuarios.
\item \textbf{RF-1.4 Listar usuarios:} el programa tiene que ser capaz de listar usuarios.
\end{itemize}
\item \textbf{RF-2 Gestión de los datos:} el programa tiene que ser capaz de gestionar los datos.
\begin{itemize}
\tightlist
\item \textbf{RF-2.1 Listar datos:} el programa tiene que ser capaz de listar el contenido de los conjuntos de datos.
\item \textbf{RF-2.2 Seleccionar datos:} el programa tiene que ser capaz de seleccionar el conjunto de dato que el usuario quiera.
\end{itemize}
\item \textbf{RF-3 Gestión de los resultados:} el programa tiene que ser capaz de gestionar los resultados.
\item \textbf{RF-4 Gestión de los modelos:} el programa tiene que ser capaz de gestionar los modelos.
\item \textbf{RF-5 Ayuda de la aplicación:} el usuario debe poder obtener ayuda sobre las funcionalidades del programa.
\end{itemize}

\subsection{Requisitos no funcionales}
\begin{itemize}
\tightlist
\item \textbf{RNF-1 Usabilidad:} la interfaz gráfica tiene que ser intuitiva y fácil de usar.
\item \textbf{RNF-2 Soporte:} el programa tiene que dar soporte a versiones iguales o mayores a Python 3.
\item \textbf{RNF-4 Localización:} el programa tiene que estar preparado para soportar varios idiomas.
\end{itemize}

\section{Especificación de requisitos}


