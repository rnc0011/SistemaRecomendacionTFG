\apendice{Documentación técnica de programación}

\section{Introducción}
En este apéndice se explica cómo está estructurado el proyecto en cuanto a los directorios, así como los pasos para compilar, instalar y ejecutar el programa.

\section{Estructura de directorios}
El proyecto está dividido en los siguientes directorios:
\begin{itemize}
\tightlist
\item \textbf{/:} carpeta raíz. Contiene la carpeta \textit{docs}, la carpeta \textit{src} y los ficheros \textit{.gitignore} y \textit{README.md}.
\item \textbf{/docs:} contiene todos los archivos relacionados con la documentación tanto en \LaTeX{} como en \textit{pdf}. También contiene los archivos de bibliografía y las imágenes que se utilizan a lo largo de la documentación.
\item \textbf{/docs/img:} carpeta que contiene las imágenes necesarias para apoyar la documentación.
\item \textbf{/docs/tex:} carpeta que contiene los distintos apartados de la memoria y los anexos en \LaTeX{}.
\item \textbf{/src:} carpeta que contiene todo el código de la aplicación.
\item \textbf{/src/controlador:} contiene los ficheros \textit{.py} con los sistemas de \textit{LightFM} y \textit{Spotlight}.
\item \textbf{/src/modelo:} contiene los ficheros \textit{.py} correspondientes a la \textit{Entrada}, \textit{Salida} y \textit{Persistencia} de los datos.
\item \textbf{/src/vista:} contiene los ficheros \textit{.py} correspondientes a la \textit{Interfaz} (de texto), \textit{Flask} (web) y \textit{Forms} (formularios de los que se compone la interfaz web). Además, contiene los \textit{.html} de la interfaz.
\item \textbf{/src/vista/templates:} contiene las páginas \textit{.html} que componen la interfaz web de la aplicación.
\item \textbf{/src/uploads:} carpeta donde se guardan los modelos y las matrices obtenidas.
\item \textbf{/src/notebooks:} esta carpeta contiene archivos que en su día se utilizaron de prueba. Se podrían borrar, pero se prefiere dejarlos para que haya constancia de que se trabajó en ellos.
\end{itemize}

\section{Manual del programador}
Las herramientas necesarias para la realización del proyecto han sido:
\begin{itemize}
\tightlist
\item Python 3.6
\item Spyder o Sublime Text
\item LightFM
\item Spotlight
\item Flask
\item WTForms
\end{itemize}
A continuación se detalla como instalar estas herramientas:

\subsection{Python}
Lenguaje de programación escogido para llevar a cabo el proyecto. Se puede descargar la última versión desde el siguiente enlace: \href{https://www.python.org/downloads/}{Descargar Python}.

\subsection{IDE}
En este caso, se ha utilizado tanto \textit{Spyder} como \textit{Sublime Text}, además de los notebooks de \textit{Jupyter}.

\textit{Spyder} y \textit{Jupyter} vienen dentro del paquete de \textit{Anaconda} (que también incluye Python). \href{https://www.datacamp.com/community/tutorials/installing-anaconda-windows}{Guía para instalar Anaconda en Windows}. En este otro enlace se descarga \textit{Anaconda} directamente para cualquier sistema operativo: \href{https://www.anaconda.com/distribution/}{Descargar Anaconda}.

\textit{Sublime Text} se puede obtener desde el siguiente enlace: \href{https://www.sublimetext.com/3}{Descargar Sublime Text}.

\subsection{LightFM}
Librería utilizada para la obtención de los modelos de recomendación clásicos. Para instalar la librería es recomendable seguir los pasos descritos en el siguiente enlace: \href{http://lyst.github.io/lightfm/docs/home.html#installation}{Instalación de LightFM}.

\subsection{Spotlight}
Librería utilizada para la obtención de los modelos de recomendación basados en aprendizaje profundo. Para instalar \textit{Spotlight} basta con ejecutar el siguiente comando: \textbf{conda install -c maciejkula -c pytorch spotlight=0.1.5} \cite{instalacion-spotlight}. Como se puede observar, con este comando también instalamos \textit{PyTorch}.

\subsection{Flask}
Con \textit{Flask} se obtiene la interfaz web. Tan solo hay que ejecutar el comando \textbf{pip install Flask} para instalarlo \cite{instalacion-flask}.

\subsection{WTForms}
Esta librería se utiliza para poder generar todos los formularios que componen la interfaz web. Para instalarla vale con ejecutar \textbf{pip install WTForms} \cite{instalacion-wtforms}.

\section{Compilación, instalación y ejecución del proyecto}
\subsection{Obtención del código}
El código se encuentra disponible en el siguiente repositorio de GitHub: \href{https://github.com/rnc0011/SistemaRecomendacionTFG}{Proyecto}. Una vez dentro del repositorio, basta con descargar el proyecto desde el botón de descarga o clonación.

\imagen{repositorio}{Descargar proyecto}

\subsection{Instalación}
Para poder compilar y ejecutar el proyecto es necesario tener instaladas todas las herramientas listadas anteriormente.

No hace falta instalar una por una las librerías y herramientas anteriores. Basta con navegar hasta el directorio raíz del proyecto y ejecutar el comando \textbf{pip install -r requirements.txt}. Esto hará que se instalen de manera automática para el usuario todas las herramientas listadas en el arhcivo \textit{requirements.txt}.

\subsection{Compilación y ejecución}
Una vez descargado el repositorio e instaladas todas las herramientas necesarias, se puede proceder a ejecutar la aplicación. Para ello, navegamos desde la consola hasta la carpeta \textit{src} del proyecto:

\imagen{navegacion_proyecto}{Navegación hasta \textit{src}}

Para ejecutar el proyecto basta con ejecutar el archivo \textit{\_\_main\_\_.py} que se encuentra dentro de \textit{src}:

\imagen{ejecucion_proyecto}{Ejecución del proyecto}

Metiendo la opción \textit{2} ejecutamos la interfaz web obtenida gracias a \textit{Flask}. Se tiene que abrir el navegador e ir a la dirección: \textbf{http://localhost:5000/home}.

\imagen{interfaz_web}{Interfaz web de la aplicación}

\section{Pruebas del sistema}
Debido a la falta de tiempo durante la realización del proyecto, no ha sido posible realizar tests sobre el código. Como ya se ha comentado en la memoria, este aspecto se deja como línea de trabajo futura.