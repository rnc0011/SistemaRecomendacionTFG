\capitulo{5}{Aspectos relevantes del desarrollo del proyecto}

Este apartado pretende recoger los aspectos más interesantes del desarrollo del proyecto.

\section{Metodologías}\label{metodologias}
Para la gestión del proyecto se ha adaptado la metodología ágil Scrum. Las principales características han sido:
 \begin{itemize}
\tightlist
\item
	La duración de los sprints fue de dos semanas aproximadamente.
\item
	Los sprints finalizaban con una reunión en la que se revisaba el trabajo hecho y se planteaban las tareas del siguiente sprint.
\item	
	Se utilizó ZenHub para llevar el seguimiento de las tareas.
\end{itemize}

\section{Formación}\label{formacion}
El proyecto ha requerido obtener una serie de conocimientos que no se tenían inicialmente. Se ha estudiado el campo de los sistemas de recomendación y la utilización de aprendizaje profundo en Python.
Para la formación en los sistemas de recomendación se usó el libro:
\begin{itemize}
\tightlist
\item
	\emph{Mining of Massive Datasets} (Jure Leskovec, Anand Rajaraman y Jeffrey D. Ullman).
\end{itemize}
Para la formación en aprendizaje profundo en Python se realizó el siguiente curso:
\begin{itemize}
\tightlist
\item
	\emph{Practical Deep Learning For Coders, Part 1} (Jeremy Howard).
\end{itemize}

\section{Desarrollo del código}\label{desarrollo-codigo}
El proyecto se ha centrado en dos aspectos fundamentales: obtener un modelo clásico y obtener un modelo basado en aprendizaje profundo.
Para la obtención del modelo clásico se estudió, en primer lugar, hacer uso de la librería Crab. Esta opción se descartó debido a la cantidad de fallos que saltaron durante la instalación (no se pudo llegar a instalar) y a la falta de actividad por parte de sus autores en el repositorio (no se ha tocado desde hace 7 años). Por todo esto, se decidió utilizar la librería LightFM.

\section{Documentación}\label{documentacion}
Las opciones que se plantearon para realizar la documentación fueron Apache OpenOffice y Texmaker. Ya que el Trabajo Fin de Grado lo veo como una forma de aprender conceptos nuevos que se desmarcan un poco de lo visto durante la carrera, opté por Texmaker debido a la novedad y al querer aprender y usar \LaTeX (es la primera vez que lo utilizo).
